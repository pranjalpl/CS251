\documentclass{article}
\usepackage[utf8]{inputenc}
\usepackage[margin=1.11in]{geometry}
\usepackage[numbers,square,sort]{natbib}
\usepackage{algorithm}
\usepackage{algpseudocode}
\usepackage[resetlabels]{multibib}
\newcites{biblio}{Bibliography}
\usepackage{amsmath}
\usepackage{flafter}
\pagenumbering{arabic}
\usepackage[english]{babel}
\usepackage{multicol}
\author{Pranjal Pratik Lal}
\date{}
\begin{document}
\title{Comparison Based Sorting Algorithms}
\maketitle
\begin{multicols}{2}
\noindent\textbf{\large Abstract}\\\\
This document presents a brief discussion on sorting
algorithms. Algorithms for \texttt{Quicksort} is provided
in this document and its working is explained.
Further, a proof of lower bounds on sorting is presented
in this document. Most of the content presented
here is created by referring and reproducing
contents from one of the widely followed book on
Algorithms by Cormen et al. \citep{1}.\textbf{We do not claim
originality of this work.} This document is prepared
as part of an assignment for the Software Lab
Course (CS251) to learn \LaTeX.\\\\
\noindent\fbox{%
    \parbox{0.48\textwidth}{%
        Declaration: I have prepared this document us-\\ing
LATEX without any unfair means. Further,\\
while the document is prepared by me, I do not\\
claim the ownership of the ideas presented in\\
this document.
    }%
}
\noindent\section{Introduction}
Sorting is one of the most fundamental operations
in computer science useful in numerous applications.
Given a sequence of numbers as input, the
output should provide a non-decreasing sequence
of numbers as output. More formally, we define a
sorting problem as follows \citep{1},\\
\textbf{Input:} A sequence of n numbers $\langle a_{1}, a_{2}, ..., a_{n} \rangle$.
\textbf{Output:} A reordered sequence (of size n)
$\langle a_{1}', a_{2}', ..., a_{n}' \rangle$ of the input sequence such that $a_{1}' \leq a_{2}' \leq . . . \leq a_{n}'$.\\
Consider the following example. Given an input
sequence h8, 34, 7, 9, 15, 91, 15i, a sorting algorithm
should return h7, 8, 9, 15, 15, 34, 91i as output.\par
A fundamental problem like sorting has attracted
many researchers who contributed with innovative
algorithms to solve the problem of sorting efficiently \citep{3}. Efficiency of an algorithm depends on
primarily on two aspects,
\begin{itemize}
\itemsep0em
\item \textbf{Time Complextity} is a formalism that captures
running time of an algorithm in terms of
the input size. Normally, asymptotic behavior
on the input size is used to analyze the time
complexity of algorithms.
\item \textbf{Space Complexity} is a formalism that captures
amount of memory used by an algorithm
in terms of input size. Like time complexity
analysis, asymptotic analysis is used for space
complexity.\\
\end{itemize}
In the branch of algorithms and complexity in computer
science, space complexity takes a back seat
compared to time complexity. Recently, another
parameter of computing i.e., energy consumption
has become popular. \citep{4} proposed an energy
complexity model for algorithms. In this document,
we will deal with time complexity of sorting
algorithms.\par
One class of algorithms which are based on element
comparison are commonly known as comparison
based sorting algorithms. In this document we
will provide a brief overview of \texttt{Quicksort}, a commonly
used comparison based sorting algorithm \citep{2}.
Quicksort is a sorting algorithm based on divide-and-conquer
paradigm of algorithm design. Further,
we will derive the lower bound of any comparison
based sorting algorithm to be $\Omega$(n log2 n)
for an input size of n.
\\
\section{Quicksort}
Quicksort is designed as a three-step divide-and-conquer
process for sorting an input sequence in
an array [1]. For any given subarray, A[i..j], the
process is as follows,\\
\textbf{Divide: } The array A[i..j] is partitioned into two
subarrays A[i..k] and A[k + 1..j] such that all elements
in A[i..k] is less than or equal to all elements
in A[k + 1..j]. A partitioning procedure is called to
determine k such that at the end of partitioning,
the element at the k
th position (i.e., A[k]) does not
change its position in the final output array.
\end{multicols}
\newpage
\begin{multicols}{2}
\noindent\makebox[\linewidth]{\rule{0.5\textwidth}{1.2pt}}\\
\noindent\textbf{Algorithm 1} Partition procedure of \texttt{Quicksort}\\algorithm.\\
\noindent\makebox[\linewidth]{\rule{0.5\textwidth}{0.4pt}}\\
\hspace{1ex} 1: \textbf{procedure} Partition(A,i,j)\\
\phantom{x}\hspace{1ex} $\triangleright$ A is an array of N integers, A[1..N]\\
\phantom{x}\hspace{1ex} $\triangleright$ i and j are the start and end of subarray\\
2: \hspace{4ex} \textit{x} $\leftarrow$ \textit{A[i]}\\
3: \hspace{4ex} \textit{y} $\leftarrow$ \textit{i-1}\\
4: \hspace{4ex} \textit{z} $\leftarrow$ \textit{j+1}\\
5: \hspace{4ex} \textbf{while} (\textit{true}) \textbf{do}\\
6: \hspace{4ex} \hspace{4ex} \textit{z} $\leftarrow$ \textit{z-1}\\
7: \hspace{4ex} \hspace{4ex} \textbf{while} (\textit{A[\textit{z}]$>$\textit{x}}) \textbf{do}\\
8: \hspace{4ex} \hspace{4ex} \hspace{4ex} \textit{z} $\leftarrow$ \textit{z-1}\\
\hspace{1ex} 9: \hspace{4ex} \hspace{4ex} \textbf{end while}\\
10:\hspace{4ex} \hspace{4ex} \textit{y} $\leftarrow$ \textit{y+1}\\
11:\hspace{4ex} \hspace{4ex} \textbf{while} \textit{A[y]$<$x} \textbf{do}\\
12:\hspace{4ex} \hspace{4ex} \hspace{4ex} \textit{y} $\leftarrow$ \textit{y+1}\\
13:\hspace{4ex} \hspace{4ex} \textbf{end while}\\
14:\hspace{4ex} \hspace{4ex} \textbf{if} \textit{y$<$z} \textbf{then}\\
15:\hspace{4ex} \hspace{4ex} \hspace{4ex} Swap \textit{A[y]} $\leftrightarrow$ \textit{A[z]}\\
16:\hspace{4ex} \hspace{4ex} \textbf{else}\\
17:\hspace{4ex} \hspace{4ex} \hspace{4ex} return \textit{z}\\
18:\hspace{4ex} \hspace{4ex} \textbf{end if}\\
19:\hspace{4ex} \textbf{end while}\\
20: \textbf{end procedure}\\
\noindent\makebox[\linewidth]{\rule{0.5\textwidth}{0.4pt}}\\
\\
\\
\textbf{Conquer:} Recursively invoke \texttt{Quicksort} on the
two subarrays. This procedure conquers the complexity by applying the same operations in two subarrays.\\\\
\textbf{Merge:} Quicksort does not require merge or combine operation as the entire array A[i::j] is sorted
in place.\par
In the heart of \texttt{Quicksort}, there is a partition
procedure as shown in Algorithm 1. A pivot element x is selected. The rst inner while loop (line\#6) continues examining elements until it nds an
element that is smaller than or equal to the pivot element. Similarly, the second inner while loop (line
\#9) continues examining elements until it nds an
element that is greater than or equal to the pivot
element. If indices y and z have not exchanged
their side around the pivot, the elements at A[y]
and A[z] are exchanged. Otherwise, the procedure
returns the index z, such that all elements to the
left of z are smaller than or equal to A[z] and all
elements to the right of z are greater than or equal
to A[z].\par
The main recursive procedure for \texttt{Quicksort} is\\\\
\noindent\makebox[\linewidth]{\rule{0.5\textwidth}{1.2pt}}\\
\noindent\textbf{Algorithm 2} \texttt{Quicksort} Recursion.\\
\noindent\makebox[\linewidth]{\rule{0.5\textwidth}{0.4pt}}\\
1:\textbf{procedure} QUICKSORT(A,i,j)\\
\hspace{4ex} $\triangleright$ Quicksort procedure called with A,1,N\\
2: \hspace{4ex} \textbf{if} i$<$j \textbf{then}\\
3: \hspace{4ex} \hspace{4ex} \textit{k} $\leftarrow$ PARTITION(\textit{A,i,j})\\
4: \hspace{4ex} \hspace{4ex} QUICKSORT(\textit{A,i,k})\\
5: \hspace{4ex} \hspace{4ex} QUICKSORT(\textit{A,k+1,j})\\
6: \hspace{4ex} \textbf{end if}\\
7: \textbf{end procedure}\\
\noindent\makebox[\linewidth]{\rule{0.5\textwidth}{0.4pt}}\\
\\
presented in Algorithm 2. Initial invocation is performed by call QUICKSORT(A; 1;N) where N is
the length of array N.

\subsection{Time Complexity analysis of \texttt{Quicksort}}
The worst case of \texttt{Quicksort} occurs when an array of length N, gets partitioned into two subarrays
of size N-1 and 1 in every recursive invocation of
QUICKSORT procedure in Algorithm 2. The partitioning procedure presented in Algorithm 1, takes
(n) time, the recurrence relation for running time
is,\\
\begin{equation*}
T(n) = T(n-1) + \Theta(n) 
\end{equation*}
As it is evident that T(1) = $\Theta$(1), the recurrence is
solved as follows,
\begin{equation*}
T(n) = T(n-1) + \Theta(n)
\end{equation*}
\begin{equation*}
=\sum_{k=1}^{n} \Theta(k)
\end{equation*}
\begin{equation*}
=\Theta(\sum_{k=1}^{n} k)
\end{equation*}
\begin{equation*}
=\Theta(n^{2})
\end{equation*}
\par
Therefore, if the partitioning is always maximally
unbalanced, the running time is $\Theta(n^{2})$. Intutively,
if an input sequence is almost sorted, \texttt{Quicksort}
will perform poorly. In the best case, partitioning
divides the array into two equal parts. Thus, the
recurrence for the best case is given by,
\begin{equation*}
=T(n) = 2T(\frac{n}{2}) + \Theta(n)
\end{equation*}
which evaluates to $\Theta(n log_{2} n)$. Using a comparatively involved analysis, the average running time
of \texttt{Quicksort} can be determined to be O(nlgn).
\end{multicols}
\newpage
\begin{multicols*}{2}
\section{Lower Bounds on comparison Sorts}
An interesting question about sorting algorithms
based on comparisons is the following: What is
the lower bound of this class of sorting algorithms? This question is important for algorithm
researchers to further improve the sorting algorithms.\par
A decision tree based analysis leads to the following theorem [1].\\\\
\textbf{Theorem 1.} \textit{Any decision tree that sorts n elements has height $\Omega(n\log_{2}n)$.}\\\\
\textit{Proof.} Consider a decision tree of height h that
sorts n elements. Since there are n! permutations
of n elements, each permutation representing a distinct sorted order, the tree must have at least n!
leaves. Since a binary tree of height h has no more
than $2^{h}$ leaves. So,
\begin{flalign*}
&n! \leq 2^{h}&
\end{flalign*}
Applying logarithmic ($\log_{2}$), the inequality becomes,\\
\begin{flalign*}
&h \geq lg(n!)&
\end{flalign*}
Applying Stirling's approximations,\\
\begin{flalign*}
&n! > (\frac{n}{e})^{n}&
\end{flalign*}
where e is natural base of logarithms. Further,\\
\begin{equation*}
h > lg(\frac{n}{e})^{n}
\end{equation*}
\begin{equation*}
= nlgn - nlge
\end{equation*}
\begin{equation*}
= \Omega(nlgn)
\end{equation*}
\section{Conclusion}
In this document, we have provided a discussion
on sorting algorithms. We included algorithms for
\texttt{Quicksort} and explained its working. Further, a
proof of lower bounds on sorting is presented in this
document. Most of the content presented here is
created by referring and reproducing contents from
one of the widely followed book on Algorithms by Cormen et al.
 \citep{1}. We do not claim originality of
this work. This document is prepared as part of an
assignment for the Software Lab Course (CS251) to
learn \LaTeX.\\
\bibliographystyle{acm}
\bibliography{one}
\end{multicols*}
\end{document}
